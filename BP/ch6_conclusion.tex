This thesis aimed to design and implement a self-hosted, open-source team scheduling system specifically suited for organizations that prioritize privacy, flexibility, and control over their scheduling data.
Through an analysis of existing solutions and their associated limitations, the project defined clear objectives:
design a user-friendly, customizable, and secure calendar application that employs modern Java web technologies and has necessary features for a team scheduling system.

The selection of Spring Boot, Thymeleaf, and PostgreSQL provided a robust foundation, effectively balancing ease of development, security, and scalability.
The implementation adhered to best practices in software architecture, using a modular MVC pattern and integrating JWT-based authentication to ensure secure stateless user sessions.
The resultant system offers a comprehensive suite of features, including advanced event management, conflict detection, filtering, and a flexible tagging system, all accessible via an intuitive Web interface.

Extensive testing at the service, controller and repository levels confirmed the reliability and correctness of the system.
The project's discussion underscored both the strengths and trade-offs of the selected technologies, as well as the challenges encountered in areas such as:
\begin{itemize}
    \item UI interactivity,
    \item Database configuration,
    \item Recurring event backend logic (frontend display incomplete).
\end{itemize}
The successful achievement of these objectives yielded several significant outcomes.
Primarily, the project facilitated a comprehensive understanding of Java web development libraries and their practical applications, realized through the evaluation, selection, and integration of technologies including Spring Boot, Thymeleaf, and PostgreSQL\@.

The outcome is a functional application designed for team work organization, enabling efficient team collaboration, providing intuitive event management, ensuring secure system interactions, and offering a responsive and user-friendly interface.

Furthermore, the project created detailed documentation covering the technology evaluation and selection process, system architecture and design decisions, implementation specifics, testing methodology, and results, which serves as an essential reference for subsequent development and analogous endeavors.

Finally, the project provided practical experience in Java web application development, database design and implementation, user interface development, and system testing and validation, enhancing skills in both technical expertise and project management.

Although the application achieves its primary objectives, certain limitations persist.
The user interface, although functional, could benefit from enhanced dynamism and modern design patterns.
Scalability for very large organizations and further automation in tag management present opportunities for future enhancement.
Additionally, easy integration with external authentication systems can enhance usability and promote broader adoption.

In conclusion, this work provides a solid technical foundation and practical insights for future development, contributing both a functional solution and a reference point for similar ventures in the domain of collaborative scheduling systems.
The project can be downloaded from the following link: \url{https://github.com/KirillBorodinskiy/TeamJob}.
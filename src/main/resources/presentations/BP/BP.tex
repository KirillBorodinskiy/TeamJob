%% LyX 2.4.3 created this file.  For more info, see https://www.lyx.org/.
%% Do not edit unless you really know what you are doing.
\documentclass[11pt,oneside,czech,american]{book}
\usepackage[T1]{fontenc}
\usepackage[utf8]{inputenc}
\setcounter{secnumdepth}{3}
\usepackage{url}
\usepackage{amsmath}
\usepackage{amsthm}
\usepackage{amssymb}
\usepackage{graphicx}
\usepackage{minted}
\usepackage[a4paper]{geometry}
\geometry{verbose,tmargin=4cm,bmargin=3cm,lmargin=3cm,rmargin=2.5cm,headheight=0.8cm,headsep=1cm,footskip=0.5cm}
\pagestyle{headings}
\usepackage{setspace}

\makeatletter
%%%%%%%%%%%%%%%%%%%%%%%%%%%%%% Textclass specific LaTeX commands.
\newenvironment{lyxlist}[1]
	{\begin{list}{}
		{\settowidth{\labelwidth}{#1}
		 \setlength{\leftmargin}{\labelwidth}
		 \addtolength{\leftmargin}{\labelsep}
		 \renewcommand{\makelabel}[1]{##1\hfil}}}
	{\end{list}}

%%%%%%%%%%%%%%%%%%%%%%%%%%%%%% User specified LaTeX commands.
%% Font setup: please leave the LyX font settings all set to 'default'
%% if you want to use any of these packages:

%% Use Times New Roman font for text and Belleek font for math
%% Please make sure that the 'esint' package is turned off in the
%% 'Math options' page.
\usepackage[varg]{txfonts}

%% Use Utopia text with Fourier-GUTenberg math
%\usepackage{fourier}

%% Bitstream Charter text with Math Design math
%\usepackage[charter]{mathdesign}

%%---------------------------------------------------------------------

%% Make the multiline figure/table captions indent so that the second
%% line "hangs" right below the first one.
%\usepackage[format=hang]{caption}

%% Indent even the first paragraph in each section
\usepackage{indentfirst}

%%---------------------------------------------------------------------

\usepackage{pdfpages}

%%---------------------------------------------------------------------

%% Disable page numbers in the TOC. LOF, LOT (TOC automatically
%% adds \thispagestyle{chapter} if not overriden
%\addtocontents{toc}{\protect\thispagestyle{empty}}
%\addtocontents{lof}{\protect\thispagestyle{empty}}
%\addtocontents{lot}{\protect\thispagestyle{empty}}

%% Shifts the top line of the TOC (not the title) 1cm upwards
%% so that the whole TOC fits on 1 page. Additional page size
%% adjustment is performed at the point where the TOC
%% is inserted.
%\addtocontents{toc}{\protect\vspace{-1cm}}

%%---------------------------------------------------------------------

% completely avoid orphans (first lines of a new paragraph on the bottom of a page)
\clubpenalty=9500

% completely avoid widows (last lines of paragraph on a new page)
\widowpenalty=9500

% disable hyphenation of acronyms
\hyphenation{CDFA HARDI HiPPIES IKEM InterTrack MEGIDDO MIMD MPFA DICOM ASCLEPIOS MedInria}

%%---------------------------------------------------------------------

%% Print out all vectors in bold type instead of printing an arrow above them
\renewcommand{\vec}[1]{\boldsymbol{#1}}

% Replace standard \cite by the parenthetical variant \citep
%\renewcommand{\cite}{\citep}

\makeatother
\bibliography{export}
\bibliographystyle{unsrt}
\usepackage{biblatex}
\usepackage{babel}
\usepackage{hyperref}
\begin{document}
\def\documentdate{August 4, 2025}

%%\def\documentdate{\today}

\pagestyle{empty}
{\centering

\noindent{}%
\begin{minipage}[c]{3cm}%
\begin{center}
\includegraphics[width=3cm,keepaspectratio,height=3cm]{Images/TITLE/cvut}
\par\end{center}%
\end{minipage}%
\begin{minipage}[c]{0.6\linewidth}%
\begin{center}
{\large\textsc{Czech Technical University in Prague}}{\large}\\
{\large Faculty of Nuclear Sciences and Physical Engineering}
\par\end{center}%
\end{minipage}%
\begin{minipage}[c]{3cm}%
\begin{center}
\includegraphics[width=3cm,keepaspectratio,height=3cm]{Images/TITLE/fjfi}
\par\end{center}%
\end{minipage}

\vspace{3cm}

{\huge\textbf{Web application for team work organization}}{\huge\par}

\vspace{1cm}

\selectlanguage{czech}%
{\huge\textbf{Webová aplikace pro organizaci týmové práce}}{\huge\par}

\selectlanguage{american}%
\vspace{2cm}

{\large Bachelor's Degree Project}{\large\par}

}

\vfill{}

\begin{lyxlist}{MMMMMMMMM}
\begin{singlespace}
\item [{Author:}] \textbf{Kirill Borodinskiy}
\item [{Supervisor:}] \textbf{doc. Ing. Miroslav Virius, CSc.}
\item [{Academic~year:}] 2024/2025
\end{singlespace}
\end{lyxlist}

\includepdf[]{BPZADANI.pdf}

\noindent{\Large\emph{Acknowledgment:}}{\Large\par}

\noindent I would like to thank my mother Elena and my girlfriend Anastasiia for their moral support.
I would like to thank my supervisor, Miroslav Virius, for their help in organization of my bachelor's thesis project.

\vfill

\noindent{\Large\emph{Author's declaration:}}{\Large\par}

\noindent I declare that this Bachelor's Degree Project is entirely
my own work and I have listed all the used sources in the bibliography.
AI tools were used in full accordance with the guidelines established
by CTU in Prague.

\bigskip{}

\noindent Prague, \documentdate\hfill{}Kirill Borodinskiy

\vspace{2cm}

\newpage{}

\selectlanguage{czech}%
\begin{onehalfspace}
\noindent\emph{Název práce:}

\noindent\textbf{Webová aplikace pro organizaci týmové práce}
\end{onehalfspace}

\bigskip{}

\noindent\emph{Autor:} Kirill Borodinskiy

\bigskip{}

\noindent\emph{Studijní program:} Celý název studijního programu
(nikoliv zkratka)\bigskip{}

\noindent\emph{Specializace:} Celý název specializace (Pokud se studijní
program nedělí na specializace, tuto řádku odstranit.)

\bigskip{}

\noindent\emph{Druh práce:} Bakalářská práce

\bigskip{}

\noindent\emph{Vedoucí práce:}doc.
Ing.
Miroslav Virius, CSc., DrSc.,
pracoviště školitele (název instituce, fakulty, katedry...)

\bigskip{}

\bigskip{}

\noindent\emph{Abstrakt:} Abstrakt max.
na 10 řádků.
Abstrakt max.
na 10 řádků.
Abstrakt max.
na 10 řádků.
Abstrakt max.
na 10 řádků.
Abstrakt max.
na 10 řádků.
Abstrakt max.
na 10 řádků.
Abstrakt max.
na 10 řádků.
Abstrakt max.
na 10 řádků.
Abstrakt max.
na 10 řádků.
Abstrakt max.
na 10 řádků.
Abstrakt max.
na 10 řádků.
Abstrakt max.
na 10 řádků.
Abstrakt max.
na 10 řádků.
Abstrakt max.
na 10 řádků.
Abstrakt max.
na 10 řádků.
Abstrakt max.
na 10 řádků.
Abstrakt max.
na 10 řádků.
Abstrakt max.
na 10 řádků.
Abstrakt max.
na 10 řádků.
Abstrakt max.
na 10 řádků.
Abstrakt max.
na 10 řádků.
Abstrakt max.
na 10 řádků.
Abstrakt max.
na 10 řádků.
Abstrakt max.
na 10 řádků.
Abstrakt max.
na 10 řádků.
Abstrakt max.
na 10 řádků.
Abstrakt max.
na 10 řádků.
Abstrakt max.
na 10 řádků.
Abstrakt max.
na 10 řádků.

\bigskip{}

\noindent\emph{Klíčová slova:} klíčová slova (nebo výrazy) seřazená
podle abecedy a oddělená čárkou

\selectlanguage{american}%
\vfill{}
~

\begin{onehalfspace}
\noindent\emph{Title:}

\noindent\textbf{Title of the Work}
\end{onehalfspace}

\bigskip{}

\noindent\emph{Author:} Kirill Borodinskiy

\bigskip{}

\noindent\emph{Abstract:} Max.
10 lines of English abstract text.
Max.
10 lines of English abstract text.
Max.
10 lines of English abstract
text.
Max.
10 lines of English abstract text.
Max.
10 lines of English
abstract text.
Max.
10 lines of English abstract text.
Max.
10 lines
of English abstract text.
Max.
10 lines of English abstract text.
Max.
10 lines of English abstract text.
Max.
10 lines of English abstract
text.
Max.
10 lines of English abstract text.
Max.
10 lines of English
abstract text.
Max.
10 lines of English abstract text.
Max.
10 lines
of English abstract text.
Max.
10 lines of English abstract text.
Max.
10 lines of English abstract text.
Max.
10 lines of English abstract
text.
Max.
10 lines of English abstract text.
Max.
10 lines of English
abstract text.
Max.
10 lines of English abstract text.
Max.
10 lines
of English abstract text.
Max.
10 lines of English abstract text.
Max.
10 lines of English abstract text.
Max.
10 lines of English abstract
text.
Max.
10 lines of English abstract text.

\bigskip{}

\noindent\emph{Key words:} keywords in alphabetical order separated
by commas

\newpage{}

\pagestyle{plain}

\tableofcontents{}

\newpage{}







\chapter{Introduction}\label{ch:introduction}
\section*{Motivation}
Efficient team schedule planning is a complex challenge, particularly in organizations that require real-time coordination and resource management.
Existing scheduling services often have significant limitations, such as proprietary nature, lack of customization, and dependence on third-party infrastructure.
This project aims to develop a self-hosted open-source booking system designed for organizations that need a private, adaptable scheduling solution.
The system will provide a web-based interface where users can:
\begin{itemize}
    \item Make and manage reservations
    \item Check real-time room availability
    \item  Filter bookings by person or room
    \item View all reservations on a centralized calendar
\end{itemize}

The backend will be built using the Spring Framework, ensuring scalability, security, and ease of integration with existing infrastructure.
Unlike cloud-based alternatives, this system will store all data locally, giving organizations complete privacy and control over scheduling information.
By combining flexibility, transparency, and data privacy, this project can provide a practical alternative to commercial scheduling tools, empowering organizations with greater autonomy and customization options.
\addcontentsline{toc}{section}{Motivation}


\pagestyle{headings}
Here we will talk about why the calendar is needed

Firstly, lets take a look at the current solution used by my university,CTU. Rozvrh.fjfi.cvut.cz is a website where students can see their schedule.
\section{The current solution}\label{sec:the-current-solution}
\begin{figure}[h]
  \centering
  \includegraphics[width=0.6\textwidth]{img}
  \caption{Screenshot of the current solution}
  \label{fig:rozvrh}
\end{figure}

It can be seen on~\ref{fig:rozvrh} that while the website completes its main purpose, it is not customizable, which makes it hard to use.
For example, if a student has a class that is from another year or/and program, they have to look at another picture and manually compare them.
From my experience, many students have screenshots on their phones and they cross-out the classes that they are not registered to.
They may have a few screenshots, for different programs or years.
It is not a good solution, as it allows for misunderstandings and mistakes.

\chapter{Methods}\label{ch:methods}
\input{methods}

\chapter*{Conclusion}

\pagestyle{plain}

\addcontentsline{toc}{chapter}{Conclusion}

Text of the conclusion\ldots{}
\begin{thebibliography}{1}
\bibitem{Allen-Cahn}S. Allen, J. W. Cahn: \emph{A microscopic theory
for antiphase boundary motion and its application to antiphase domain
coarsening}.
Acta Metall., 27:1084-1095, 1979.

\bibitem{CINECA}G. Ballabio et al.: \emph{High Performance Systems
User Guide}.
High Performance Systems Department, CINECA, Bologna,
2005. \url{www.cineca.it}

\bibitem{rumpf3}J. Becker, T. Preusser, M. Rumpf: \emph{PDE methods
in flow simulation post processing}.
Computing and Visualization in
Science, 3(3):159-167, 2000.

\end{thebibliography}

\end{document}
